\documentclass[usenames,dvipsnames, 8pt]{beamer}
\mode<presentation> {


\usetheme{Madrid}
\usefonttheme{serif}

}



\usepackage{amsfonts}
\usepackage{amsthm}
\usepackage{amssymb}
\usepackage{slashed}
\usepackage{bm}
\usepackage{amsmath, mathtools}
\usepackage[compat=1.0.0]{tikz-feynman}

\newcommand{\feyn}[1]{\begin{gathered}\begin{tikzpicture}\begin{feynman}[scale=1.5] #1 \end{feynman}\end{tikzpicture}\end{gathered}}


\setbeamertemplate{navigation symbols}{}

\title[Main Topic Here]{Main Topic } 

\author{Your Name} 
\institute[Your Institution] 
{
University of InertialObservr \\ 
\medskip
\textit{email} % Your email address
}
\date{\today} % Date, can be changed to a custom date


\begin{document}

\begin{frame}
\frametitle{Feynman Diagram}



\begin{align*}
%
&m_h^2 = \\
& \feyn{
\node [blob] (center) at (0,0) ;
\vertex (x1) at (-1,0) {$\phi$};
\vertex (p1) at (1,0) {$\phi$};
%
%
\diagram *{
(x1) -- [scalar] (center) ;
(center) -- [scalar] (p1);
};
}
= 
\feyn{
%\node [blob] (center) at (0,0) ;
\vertex (x1) at (-1,0) {$\phi$};
\vertex (p1) at (0,0) {$\phi$};
%
%
\diagram *{
(x1) -- [scalar] (p1);
};
}
\ \ + \ \ 
\sum_\psi
\feyn{
%\node [blob] (center) at (0,0) ;
\vertex (x1) at (-1,0) {$\phi$};
\vertex (lv) at (-.3,0);
\vertex (rv) at (.3,0);
\vertex (p1) at (1,0) {$\phi$};
%
%
\diagram *{
(x1) -- [scalar] (lv);
(lv) -- [half left, fermion,edge label'=\(\psi\)] (rv);
(lv) -- [half right, anti fermion] (rv);
(rv) -- [scalar] (p1);
};
}
\\
&= (m^0_h)^2 + \Delta m_h^2
%
\end{align*}



\end{frame}


\end{document}
